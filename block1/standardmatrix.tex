

\beginedxvertical{Page One}

\beginedxtext{Preliminaries}


For time budgeting purposes, this sequence has 4 videos totaling 14 minutes, 
plus some questions.  

% Remember, when you're doing the online learning sequences, you may seek help if you 
% do not understand a video, but you should think about all of the questions 
% entirely individually.  You have pledged to do so under the Honor Code!  


\endedxtext

\endedxvertical



\beginedxvertical{Rotation Example}


\beginedxproblem{Rotate a Vector}{\dpa1}

Let $T: \R^2 \rightarrow \R^2$ be the function that rotates vectors about the origin counterclockwise by
$45^\circ$.  

What is  $T\left( \left[\begin{array}{c}
1 \\
0 
\end{array} \right] \right)$?  




\endedxproblem

\beginedxproblem{Rotate another Vector}{\dpa1}


What is $T\left( \left[\begin{array}{c}
0 \\
1 
\end{array} \right] \right)$?  


\endedxproblem


\endedxvertical





\beginedxvertical{Rotation Example Continued}



\beginedxproblem{More Rotation}{\dpa1}

We'll continue working with the same function $T$ which rotates vectors in $\R^2$ counterclockwise by $45^\circ$.  Previously, we found that $T(\left[\begin{array}{c}
1 \\
0 
\end{array} \right]) = \left[\begin{array}{c}
\frac{\sqrt{2}}{2} \\
\frac{\sqrt{2}}{2} 
\end{array} \right] \approx  \left[\begin{array}{c}
0.71 \\
0.71 
\end{array} \right]
,$ and 
$T(\left[\begin{array}{c}
0 \\
1 
\end{array} \right]) = \left[\begin{array}{c}
-\frac{\sqrt{2}}{2} \\
\frac{\sqrt{2}}{2} 
\end{array} \right] \approx  \left[\begin{array}{c}
-0.71 \\
0.71 
\end{array} \right]
 . $ 
 
Based on those results, what do you think $T(\left[\begin{array}{c}
4 \\
0 
\end{array} \right]$ will be?





\edXsolution{ 
}

\endedxproblem





\beginedxproblem{Last Rotation}{\dpa1}

Finally, use more geometric reasoning to say what $T(\left[\begin{array}{c}
4 \\
0 
\end{array} \right]$ is.  





\edXsolution{ 
}

\endedxproblem



\doedxvideo{Standard Matrix for a Transformation}{???}


\endedxvertical

