

\usepackage{enumerate}

\beginedxvertical{Page One}

\beginedxtext{Preliminaries}





At the end of this sequence, and after some practice, you should be able to:

\begin{itemize}
\item Understand the definition of linear independence.  
\item Be able to determine whether a list of vectors is linearly independent. 
\item Know conditions on whether a list of vectors in $\R^n$ can be linearly independent.
\end{itemize}


For time budgeting purposes, this sequence has X videos totaling X minutes, 
plus some questions.  




\endedxtext

\endedxvertical


\beginedxvertical{Linear Independence}


\beginedxvertical{Linear Independence Problems}


\beginedxtext{Definitions of Linearly Independent and Dependent}

{\keya{\bf{Definition.}}} We say that a list of vectors $\{v_1; v_2; \ldots v_n\}$ in a vector space $V$ is 
{\keyb{\bf linearly independent}} if the only way to pick scalars $a_1, a_2, \ldots a_n$ such that
$a_1v_1 + a_2v_2 + \ldots + a_nv_n = \veco$ is if $a_1 = a_2 = \ldots = a_n = 0$.  

{\keya{\bf{Definition.}}} A list of vectors $\{v_1; v_2; \ldots v_n\}$ in a vector space $V$ is 
{\keyb{\bf linearly dependent}} if there are scalars $a_1, a_2, \ldots a_n$, not all zero, such that
$a_1v_1 + a_2v_2 + \ldots + a_nv_n = \veco$.  


\endedxtext

\beginedxproblem{List 1}{\dpa5}

\begin{edXscript}

def test_ld_1(expect,ans):
    import ast
    import numpy as np 
    if (ans=='No') or (ans=='NO') or (ans=='no'):
        return {'ok': False, 'msg': 'The answer is not no'}
    list_expect = ast.literal_eval('[[1,-1,0,0],[2,-1,3,5],[-3,3,0,0]]')
    list_vec = []
    for vec in list_expect:
        list_vec.append(np.array(vec))
    list_scalars = [float(s) for s in ans.split(',')]
    ret = {'ok':False}
    # check scalars are not all zero
    max_sca = max(list_scalars)
    min_sca = min(list_scalars)
    if (max_sca == 0) and (min_sca ==0):
        ret['msg'] = 'Your answer should contain at least one non-zero scalar.'
        return ret
    # check there are enough scalars
    if (len(list_scalars) != len(list_vec)):
        ret['msg'] = 'Your answer contains too many/too few scalars.'
        return ret    
    # check if a0*v0 + a1*v1 + ... + a_n*vn = 0
    n = len(list_scalars)
    sum = 0
    for i in range(0,n):
        sum += list_scalars[i] * list_vec[i]
    print(sum)
    if np.allclose(sum, np.zeros(4)):
        ret['ok'] = True
    return ret  
\end{edXscript}


Consider the list of vectors $\{v_1; v_2; v_3\}$, where

\[v_1 = \left[\begin{array}{c} 1 \\ -1  \\ 0 \\ 0 \end{array} \right], 
v_2 = \left[\begin{array}{c} 2 \\ -1  \\ 3 \\ 5 \end{array} \right],  
v_3 = \left[\begin{array}{c} -3 \\ 3  \\ 0 \\ 0 \end{array} \right]. \]

Can you find scalars $a_1, a_2, a_3$, not all zero, such that 
$a_1v_1 + a_2 v_2 + a_3 v_3 = \veco$?  If so, enter an example of such scalars below.  (Use decimals only, separated by commas.  For instance, enter 3,2,5 if you want $a_1 = 3, a_2=2, a_3=5$.)
If not, enter `No'.



\edXabox{type="custom" cfn="test_ld_1" expect="3,0,1"}


Is the list $\{v_1; v_2; v_3\}$ linearly dependent or linearly independent?  

\edXabox{expect="Linearly dependent" options="Linearly dependent","Linearly independent"}


\edXsolution{ 
We notice that $-3v_1 = v_3$, so that $3  v_1 + 0 v_2 + 1  v_3 = \veco$. Therefore, $a_1 = 3$, $a_2 = 0$, and $a_3 = 1$ is a possible set of scalars that satisfies the equation. (Any scalar multiple of that triple is also a set that satisfies the equation.)

The list $\{v_1; v_2; v_3\}$ is linearly dependent because, as we've just seen, there exist sets of coefficients, not all zero, that yield $a_1v_1 + a_2 v_2 + a_3 v_3 = \veco$.
}


\endedxproblem



\beginedxproblem{List 2}{\dpa5}


\begin{edXscript}

def test_ld_12(expect,ans):
    import ast
    import numpy as np 
    if (ans=='No') or (ans=='NO') or (ans=='no'):
        return {'ok': False, 'msg': 'The answer is not no'}
    list_expect = ast.literal_eval('[[1,-1,3,4,5],[0,0,0,0,0],[-3,3,3,5,1]]')
    list_vec = []
    for vec in list_expect:
        list_vec.append(np.array(vec))
    list_scalars = [float(s) for s in ans.split(',')]
    ret = {'ok':False}
    # check scalars are not all zero
    max_sca = max(list_scalars)
    min_sca = min(list_scalars)
    if (max_sca == 0) and (min_sca ==0):
        ret['msg'] = 'Your answer should contain at least one non-zero scalar.'
        return ret
    # check there are enough scalars
    if (len(list_scalars) != len(list_vec)):
        ret['msg'] = 'Your answer contains too many/too few scalars.'
        return ret    
    # check if a0*v0 + a1*v1 + ... + a_n*vn = 0
    n = len(list_scalars)
    sum = 0
    for i in range(0,n):
        sum += list_scalars[i] * list_vec[i]
    print(sum)
    if np.allclose(sum, np.zeros(5)):
        ret['ok'] = True
    return ret  
\end{edXscript}

Consider the list of vectors $\{v_1; v_2; v_3\}$, where

\[v_1 = \left[\begin{array}{c} 1 \\ -1  \\ 3 \\ 4 \\5 \end{array} \right], 
v_2 = \left[\begin{array}{c} 0 \\ 0  \\ 0 \\ 0 \\ 0 \end{array} \right],  
v_3 = \left[\begin{array}{c} -3 \\ 3  \\ 3 \\ 5 \\ 1 \end{array} \right]. \]

Can you find scalars $a_1, a_2, a_3$, not all zero, such that 
$a_1v_1 + a_2 v_2 + a_3 v_3 = \veco$?  If so, enter an example of such scalars below.  (Use decimals only, separated by commas.  For instance, enter 3,2,5 if you want $a_1 = 3, a_2=2, a_3=5$.)
If not, enter `No'.  

\edXabox{type="custom" cfn="test_ld_12" expect="0,1,0"}


Is the list $\{v_1; v_2; v_3\}$ linearly dependent or linearly independent?  

\edXabox{expect="Linearly dependent" options="Linearly dependent","Linearly independent"}

\edXsolution{ The first answer is immediately "yes" because we have the zero vector in our set; we can choose any non-zero value for $a_2$, and with $a_1 = 0$ and $a_3 = 0$, the above equation will still be true. Because there exist sets of coefficients, not all zero, that make the above equation true, the list $\{v_1; v_2; v_3\}$ is linearly dependent.
}

\endedxproblem






\endedxvertical



