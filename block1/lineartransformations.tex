

\beginedxvertical{Page One}

\beginedxtext{Preliminaries}


At the end of this sequence, and after some practice, you should be able to:

\begin{itemize}
\item Understand the definition of linear transformation.  
\item Be able to determine whether a function between vector spaces is a linear transformations. 
\item Recognize basic properties of linear transformations.
\end{itemize}

For time budgeting purposes, this sequence has 4 videos totaling 14 minutes, 
plus some questions.  

% Remember, when you're doing the online learning sequences, you may seek help if you 
% do not understand a video, but you should think about all of the questions 
% entirely individually.  You have pledged to do so under the Honor Code!  


\endedxtext

\endedxvertical

\beginedxvertical{Introduction}



\doedxvideo{Functions}{nMrAAozL4F0}


\beginedxproblem{System Consistency?}{\dpa1}

Define the function $T$ by $T(\vecx) = A\vecx$, where $A = \left[ \begin{array}{ccc}
6 & 5 & 3 \\
2 & 4 &  23 
\end{array} \right].$

What is the domain of $T$?

\edXabox{type="multichoice" expect="$\R^3$" options="$\R$","$\R^2$","$\R^3$"}

What is the codomain of $T$?

\edXabox{type="multichoice" expect="$\R^2$" options="$\R$","$\R^2$","$\R^3$"}

\edXsolution{ 

}

\endedxproblem



\endedxvertical





\beginedxvertical{Defining Linear Transformations}

\doedxvideo{Linear Transformations}{ASQadpJlL94}
%example: rotations, non-example: something quadratic


\endedxvertical









\beginedxvertical{Linear Transformation Definition}

\beginedxtext{Definition of Linear Transformation}

Let $\F$ be a field.  A function $T: \F^n \rightarrow \F^m$ is a {\keyb{\bf linear transformation}}
if both of the following properties hold:

\begin{itemize}
\item For any vectors $\vecv, \vecw$ in the domain, $T(\vecv + \vecw) = T(\vecv) + T(\vecw).$ 
\item For any vector $\vecv$ in the domain, and any scalar $a\in \F$, $T(a\vecv) = aT(\vecv)$.    
\end{itemize}

Informally, if $T$ satisfies the first condition we say that ``$T$ respects vector addition", and 
if $T$ satisfies the second condition we say that ``$T$ respects scalar multiplication."  

\endedxtext

\endedxvertical


\beginedxvertical{Some Examples and Non-examples}



\beginedxproblem{Linear Transformation? 1}{\dpa1}

Define $T: \R^3 \rightarrow \R^3$ by $T(\vecv) = 2\vecv$.  

Does $T$ respect vector addition?  That is, is $T(\vecv + \vecw) = T(\vecv) + T(\vecw)$ for all 
vectors $\vecv,\vecw$?
Try a couple of specific numerical examples if you're not sure.  

\edXabox{expect="Yes" options="Yes","No"}

Does $T$ respect scalar multiplication?  
That is, is $T(a\vecv) = aT(\vecv)$ for all vectors $\vecv$ and scalars $a$?
Try a couple of specific numerical examples if you're not sure.  


\edXabox{expect="Yes" options="Yes","No"}

Is $T$ a linear transformation?

\edXabox{expect="Yes" options="Yes","No"}

\edXsolution{ 
}

\endedxproblem


\beginedxproblem{Linear Transformation? 2}{\dpa1}

Define $T: \R^2 \rightarrow \R^2$ by $T(\vecv) = \veco$.  

Does $T$ respect vector addition?  That is, is $T(\vecv + \vecw) = T(\vecv) + T(\vecw)$ for all 
vectors $\vecv,\vecw$?
Try a couple of specific numerical examples if you're not sure.  

\edXabox{expect="Yes" options="Yes","No"}

Does $T$ respect scalar multiplication?  
That is, is $T(a\vecv) = aT(\vecv)$ for all vectors $\vecv$ and scalars $a$?
Try a couple of specific numerical examples if you're not sure.  


\edXabox{expect="Yes" options="Yes","No"}

Is $T$ a linear transformation?

\edXabox{expect="Yes" options="Yes","No"}

\edXsolution{ 
}

\endedxproblem

\beginedxproblem{Linear Transformation? 3}{\dpa1}

Define $T: \R^2 \rightarrow \R^2$ by $T(\vecv) = \left[\begin{array}{c}
4 \\
0 
\end{array} \right]$.  

Does $T$ respect vector addition?  That is, is $T(\vecv + \vecw) = T(\vecv) + T(\vecw)$ for all 
vectors $\vecv,\vecw$?
Try a couple of specific numerical examples if you're not sure.  

\edXabox{expect="No" options="Yes","No"}

Does $T$ respect scalar multiplication?  
That is, is $T(a\vecv) = aT(\vecv)$ for all vectors $\vecv$ and scalars $a$?
Try a couple of specific numerical examples if you're not sure.  


\edXabox{expect="No" options="Yes","No"}

Is $T$ a linear transformation?

\edXabox{expect="No" options="Yes","No"}

\edXsolution{ 
}

\endedxproblem


\endedxvertical


\beginedxvertical{More Linear Transformations}

\doedxvideo{A Linear Transformation outside of R^n}{Pyv9GXNYJ44}



\beginedxproblem{Linear Transformation? 4}{\dpa1}

Define $T: \mathbb{P} \rightarrow \R$ by $T(f) = f(1)$.  

To get a sense of how $T$ works, let's do a quick example.  Suppose $f$ is the element of 
$\mathbb{P}$ given by $f(t) = 2t^2 -5$.  
What is $T(f)$?  

\edXabox{type="numerical" expect="-3"}


Does $T$ respect vector addition?  That is, is $T(f+g) = T(f) + T(g)$ for all 
vectors $f,g \in \mathbb{P}$?
Try a couple of specific numerical examples if you're not sure.  

\edXabox{expect="Yes" options="Yes","No"}

Does $T$ respect scalar multiplication?  
That is, is $T(af) = aT(f)$ for all vectors $f \in \mathbb{P}$ and scalars $a$?
Try a couple of specific numerical examples if you're not sure.  


\edXabox{expect=Yes" options="Yes","No"}

Is $T$ a linear transformation?

\edXabox{expect="Yes" options="Yes","No"}

\edXsolution{ 
}

\endedxproblem



\endedxvertical


\beginedxvertical{Linear Transformations Properties}

\doedxvideo{Linear Transformation Properties}{oe4fgdPAsgA}




\beginedxproblem{Linear Transformation Practice}{\dpa1}

Suppose $T: \mathbb{P} \rightarrow \R^2$ is a linear transformation.  Let $f,g\mathbb{P}$ be
the polynomials given by $f(t) = t^2$ and $g(t) = t$.  

Suppose that $T(f) = \left[\begin{array}{c} 1 \\ 2  \end{array} \right]$
and $T(g) = \left[\begin{array}{c} 2 \\ -1  \end{array} \right]$



If $h \in \mathbb{P}$ is the polynomial given by $h(t) = -2t^2$, what must $T(h)$ be?

To enter a vector such as $\left[\begin{array}{c} 1 \\ 2  \\ 3 \end{array} \right]$, you can either:
\begin{itemize}
\item
enter it as 
you would a $3\times 1$ matrix: [[1],[2],[3]]  
\item
or, you can enter it as <1,2,3>
\end{itemize}

Use decimals only.  

\begin{edXscript}
def VectorEntry(expect, ans):
	import ast
	import numpy as np 
  	atol = 0.01
	list_expect = ast.literal_eval(expect)
	vec_expect = np.matrix(list_expect)
  	ret = {"ok":False}
	try:
  		# input format [[1],[2],[3]]
		list_ans = ast.literal_eval(ans)
		vec_ans = np.matrix(list_ans)
  		if vec_ans.shape != vec_expect.shape:
  			ret['msg'] = 'Wrong shape of vector!'
  		elif np.allclose(vec_ans, vec_expect,atol,1e-08):
  			ret['ok'] = True
  		else:
  		# More error message. Will improve this part
  			ret['msg'] = 'something is wrong'   			
	#except SyntaxError:
		#ret['msg'] = 'Wrong input format'
	except SyntaxError:
  		# input format &lt;1,2,3&gt;
		list_ans = ans.replace('&lt;', '[').replace('&gt;', ']')
		list_ans = ast.literal_eval(list_ans)
		vec_ans = np.transpose(np.matrix(list_ans))
  		if vec_ans.shape != vec_expect.shape:
  			ret['msg'] = 'Wrong shape of vector!'
  		elif np.allclose(vec_ans, vec_expect,0.01,1e-08):
  			ret['ok'] = True
  		else:
    		# More error message. Will improve this part
  			ret['msg'] = 'something is wrong' 
  	except:
  		ret['msg'] = 'Wrong input format'
  	return ret 
\end{edXscript}



\edXabox{type="custom" cfn="VectorEntry" expect="[[-2],[-4]]"}


If $j \in \mathbb{P}$ is the polynomial given by $h(t) = 3t^2-t$, what must $T(j)$ be?



\edXabox{type="custom" cfn="VectorEntry" expect="[[1],[7]]"}


\edXsolution{ 
}


\endedxproblem



\endedxvertical


% \beginedxvertical{Linear Transformations Properties}

% \doedxvideo{Matrix Multiplication as a Linear Transformation}{jMRJ-efZOJs}



% \endedxvertical





