

\beginedxvertical{Page One}

\beginedxtext{Preliminaries}





At the end of this sequence, and after some practice, you should be able to:

\begin{itemize}
\item Understand the definition of linear independence.  
\item Be able to determine whether a list of vectors is linearly independent. 
\item Know conditions on whether a list of vectors in $\R^n$ can be linearly independent.
\end{itemize}


For time budgeting purposes, this sequence has X videos totaling X minutes, 
plus some questions.  




\endedxtext

\endedxvertical


\beginedxvertical{Linear Independence}


\beginedxproblem{Existence vs. Uniqueness}{\dpa2}

Given a matrix $A$, consider the equation $A\vecx = \veco$.   

Regarding such equations, we have discussed two main questions: the existence question
(does there exist at least one solution?) and the uniqueness question (is there at most one
solution?).  For this equation, one of these questions always has the same answer, no matter
what $A$ is.  Which question does not always have the same answer?  

\edXabox{expect="Uniqueness" options="Existence","Uniqueness"}

\edXsolution{  }

\endedxproblem

\doedxvideo{Linear Independence}{jQSgCZwOKQU}


\beginedxproblem{Definition}{\dpa1}


Which of these is the definition of linearly dependent?  

A list of vectors $\{v_1; v_2; \ldots v_n\}$ is linearly dependent if...

\edXabox{type="multichoice" expect="...there exist scalars
$a_1, a_2, \ldots a_n$, not all zero, such that $a_1v_1 + a_2v_2 + \ldots a_nv_n = \veco$." options="...there exist scalars
$a_1, a_2, \ldots a_n$, not all zero, such that $a_1v_1 + a_2v_2 + \ldots a_nv_n = \veco$.","...there exist scalars
$a_1, a_2, \ldots a_n$, all not zero, such that $a_1v_1 + a_2v_2 + \ldots a_nv_n = \veco$.","Both of these are the definition."}

\edXsolution{ 
}


\endedxproblem


\endedxvertical


\beginedxvertical{Linear Independence Problems}


\beginedxtext{Definitions of Linearly Independent and Dependent}

We say that a list of vectors $\{v_1; v_2; \ldots v_n\}$ in a vector space $V$ is 
{\keyb{\bf linearly independent}} if the only way to pick scalars $a_1, a_2, \ldots a_n$ such that
$a_1v_1 + a_2v_2 + \ldots + a_nv_n = \veco$ is if $a_1 = a_2 = \ldots = a_n = 0$.  

A list of vectors $\{v_1; v_2; \ldots v_n\}$ in a vector space $V$ is 
{\keyb{\bf linearly dependent}} if there are scalars $a_1, a_2, \ldots a_n$, not all zero, such that
$a_1v_1 + a_2v_2 + \ldots + a_nv_n = \veco$.  


\endedxtext

\beginedxproblem{List 1}{\dpa1}


Consider the list of vectors $\{v_1; v_2; v_3\}$, where

\[v_1 = \left[\begin{array}{c} 1 \\ -1  \\ 0 \\ 0 \end{array} \right], 
v_2 = \left[\begin{array}{c} 2 \\ -1  \\ 3 \\ 5 \end{array} \right],  
v_3 = \left[\begin{array}{c} -3 \\ 3  \\ 0 \\ 0 \end{array} \right]. \]

Can you find scalars $a_1, a_2, a_3$, not all zero, such that 
$a_1v_1 + a_2 v_2 + a_3 v_3 = \veco$?  If so, enter an example of such scalars below.
If not, enter `No'  

\edXabox{expect="Placeholder" options="Placeholder"}



Is the list $\{v_1; v_2; v_3\}$ linearly dependent or linearly independent?  

\edXabox{expect="Linearly dependent" options="Linearly dependent","Linearly independent"}


\edXsolution{ 
}


\endedxproblem



\beginedxproblem{List 2}{\dpa1}


Consider the list of vectors $\{v_1; v_2; v_3\}$, where

\[v_1 = \left[\begin{array}{c} 1 \\ -1  \\ 3 \\ 4 \\5 \end{array} \right], 
v_2 = \left[\begin{array}{c} 0 \\ 0  \\ 0 \\ 0 \\ 0 \end{array} \right],  
v_3 = \left[\begin{array}{c} -3 \\ 3  \\ 3 \\ 5 \\ 1 \end{array} \right]. \]

Can you find scalars $a_1, a_2, a_3$, not all zero, such that 
$a_1v_1 + a_2 v_2 + a_3 v_3 = \veco$?  If so, enter an example of such scalars below.
If not, enter `No'.  

\edXabox{expect="Placeholder" options="Placeholder"}


Is the list $\{v_1; v_2; v_3\}$ linearly dependent or linearly independent?  

\edXabox{expect="Linearly dependent" options="Linearly dependent","Linearly independent"}

\edXsolution{ 
}


\endedxproblem




\beginedxproblem{List 3}{\dpa1}


Let $\mathbb{P}$ be the vector space of polynomials with real coefficients in the variable $t$.  
Consider the list of vectors $\{f; g\}$, where $f(t) = 1+t$ and $g(t) = t+t^2$ are
vectors in $\mathbb{P}$.

Can you find scalars $a_1, a_2$, not both zero, such that 
$a_1f + a_2 g = \veco$?  If so, enter an example of such scalars below.
If not, type `No'.  

\edXabox{expect="Placeholder" options="Placeholder"}


Is the list $\{v_1; v_2\}$ linearly dependent or linearly independent?  

\edXabox{expect="Linearly independent" options="Linearly dependent","Linearly independent"}

\edXsolution{ 
}


\endedxproblem



\endedxvertical




\beginedxvertical{Geometry of Linear Independence}


\doedxvideo{Linear Independence for Small Sets}{Mw04hBXVtho}


\beginedxproblem{More Lists}{\dpa3}


Consider the vectors $v_1, v_2, v_3 \in \R^2$ as given in this picture:


\begin{center}
\includesvg[300]{c1s6threevectors}   
\end{center}

Which of the following lists are linearly independent?  Click all that apply.  


\edXabox{type="oldmultichoice" expect="$\{v_1\}$","$\{v_1;v_2\}$","$\{v_2;v_3\}$" options="$\{\veco\}$","$\{v_1\}$","$\{v_1;v_2\}$","$\{v_1;v_3\}$","$\{v_2;v_3\}$","$\{v_3;v_3\}$","$\{v_1; v_2; v_3\}$"}


\edXsolution{ 
}


\endedxproblem



\doedxvideo{A Proposition}{NMB5tCmE2UE}

\endedxvertical

\beginedxvertical{Linear Dependence Proposition}

\beginedxtext{Linear Dependence Equivalent Condition}

Let $\{v_1; v_2; \ldots v_n\}$ be vectors in a vector space $V$.  The list is linearly dependent
if and only if one of the vectors in the list can be expressed as a linear combination of the other 
vectors.  

\endedxtext

\beginedxproblem{True or False}{\dpa1}


True or False: Suppose the list $\{v_1; v_2; v_3; v_4\}$ is linearly dependent.  Then $v_4$ must be in the 
span of $\{v_1; v_2; v_3\}$.  

\edXabox{expect="False" options="True","False"}


\edXsolution{ 
}


\endedxproblem


\endedxvertical

\beginedxvertical{Determining Linear Independence in R^m}



\beginedxproblem{Proving Linear Independence}{\dpa3}

Let's think about the definition of linear independence.  
Recall that a list of vectors $\{v_1; v_2; \ldots v_n\}$ in a vector space $V$ is 
linearly independent if the only way to pick scalars $a_1, a_2, \ldots a_n$ such that
$a_1v_1 + a_2v_2 + \ldots + a_nv_n = \veco$ is if $a_1 = a_2 = \ldots = a_n = 0$.  


Which of the following would be a way to prove that a list $\{v_1; v_2; \ldots v_n\}$ is linearly independent?
Check all that apply.

\edXabox{type="oldmultichoice" expect="Assume that $a_i \ne 0$ for at least one $i$ from 1 to $n$, and show $a_1v_1 + a_2v_2 + \ldots + a_nv_n \ne \veco$","Assume that $a_1v_1 + a_2v_2 + \ldots + a_nv_n = \veco$, and show that $a_i = 0$ for all $i$ from 1 to $n$" options="Assume that $a_i = 0$ for all $i$ from 1 to $n$, and show $a_1v_1 + a_2v_2 + \ldots + a_nv_n = \veco$","Assume that $a_i \ne 0$ for all $i$ from 1 to $n$, and show $a_1v_1 + a_2v_2 + \ldots + a_nv_n \ne \veco$","Assume that $a_i \ne 0$ for at least one $i$ from 1 to $n$, and show $a_1v_1 + a_2v_2 + \ldots + a_nv_n \ne \veco$","Assume that $a_1v_1 + a_2v_2 + \ldots + a_nv_n = \veco$, and show that $a_i = 0$ for all $i$ from 1 to $n$","Assume that $a_1v_1 + a_2v_2 + \ldots + a_nv_n \ne \veco$, and show that $a_i \ne 0$ for at least one $i$ from 1 to $n$"}


\edXsolution{ 
}


\endedxproblem



\beginedxproblem{Non-trivial Solutions?}{\dpa2}

Use row reduction to determine how many solutions the equation $Ax = \veco$ has, where 
\[ A = \left[\begin{array}{ccc} 1 & 2 & 3 \\ -1 & 1 & 3 \\ 3 & 3 & 3 \\ 4 & 1 & -2 \\
5  & 3 & 1 \end{array} \right]. \]

\edXabox{expect="Infinitely many solutions" options="No solutions","Exactly one solution","Infinitely many
solutions"}



Is the list $\{v_1; v_2; v_3\}$ linearly independent, where the $v_i$ are given below?

\[v_1 = \left[\begin{array}{c} 1 \\ -1  \\ 3 \\ 4 \\5 \end{array} \right], 
v_2 = \left[\begin{array}{c} 2 \\ 1  \\ 3 \\ 1 \\ 3 \end{array} \right],  
v_3 = \left[\begin{array}{c} 3 \\ 3  \\ 3 \\ -2 \\ 1 \end{array} \right]. \]


\edXabox{expect="Linearly dependent" options="Linearly independent","Linearly dependent"}

\edXsolution{ 
}
\endedxproblem





\beginedxproblem{Fill in the blank}{\dpa2}

What goes in the blank to make a correct statement?

If the matrix $A$ when row-reduced yields $\underline{\;\;\;\;\;\;\;\;\;\;}$, then
the columns of $A$ cannot be linearly independent.  

\edXabox{type="multichoice" expect="a free variable" options="a row of all zeros","a free variable","either of the above"}

\edXsolution{ 
}
\endedxproblem


\beginedxproblem{Fill in the blank 2}{\dpa2}

What goes in the blank to make a correct statement?

If the matrix $A$ has $\underline{\;\;\;\;\;\;\;\;\;\;}$, then
there must be a free variable.  

\edXabox{type="multichoice" expect="more columns than rows" options="more rows than columns","more columns than rows"}

\edXsolution{ 
}
\endedxproblem


\beginedxproblem{Fill in the blank 3}{\dpa2}

Putting the previous two statements together, what goes in the blank to make a correct statement?  

A list of  $\underline{\;\;\;\;\;\;\;\;\;\;}$ cannot be linearly independent.

\edXabox{type="multichoice" expect="5 vectors in $\R^3$" options="3 vectors in $\R^5$","5 vectors in $\R^3$"}

\edXsolution{ 
}
\endedxproblem


\doedxvideo{Linear Independence in R^m}{???}


\endedxvertical

\beginedxvertical{Linearly Independent Lists in R^m}


\beginedxtext{Linear Independence in R^m}


We have just shown the following results:  

If $A$ is an $m\times n$ matrix with columns 
$v_1, v_2, \ldots v_n \in \R^m$, then the list $\{v_1; v_2; \ldots v_n\}$ is linearly independent
if and only if $A$ does not have a free variable.  

If $n>m$, then any list of $n$ vectors in $\R^m$ must be linearly dependent.  


\endedxtext





\beginedxproblem{Quick Picks}{\dpa2}

Without doing any row-reduction, you should be able to pick out three of the 
following lists of vectors that are linearly dependent.  Which three?  

\begin{enumerate}


\item
$\left\{ \left[ \begin{array}{c} 1 \\ 1 \\ 1 \end{array} \right]; 
\left[ \begin{array}{c} 1 \\ 1 \\ 0 \end{array} \right]; 
\left[ \begin{array}{c} 1 \\ 0 \\ 1 \end{array} \right] \right\}
$


\item
$\left\{\left[ \begin{array}{c} 1 \\ 2 \\ 3 \end{array} \right] ; 
\left[ \begin{array}{c} 0 \\ 0 \\ 0 \end{array} \right] ; 
\left[ \begin{array}{c} 2 \\ 5 \\ 7 \end{array} \right]\} $



\item
$\left\{\left[ \begin{array}{c} 1 \\ 2 \\ 1 \\ 4\end{array} \right] ; 
\left[ \begin{array}{c} 2 \\ 13 \\ 3 \\ 7\end{array} \right] ; 
\left[ \begin{array}{c} 0 \\ 1 \\ 5 \\ -3\end{array} \right] ;
\left[ \begin{array}{c} 15 \\ -2 \\ 6 \\ 2\end{array} \right] ;
\left[ \begin{array}{c} 5 \\ 3 \\ 8 \\ 11\end{array} \right] \right\} $



\item
$\left\{\left[ \begin{array}{c} 2 \\ 3 \\ 5 \\ 7 \\ 11 \end{array} \right] ;
\left[ \begin{array}{c} 1 \\ 2 \\ 3 \\ 4 \\5 \end{array} \right] ;
\left[ \begin{array}{c} 2 \\ 4 \\ 6 \\ 8 \\ 10 \end{array} \right] ; 
\left[ \begin{array}{c} 1 \\ 1 \\ 2 \\ 3 \\ 5 \end{array} \right] \right\} $

\item
$\left\{\left[ \begin{array}{c} 0 \\ 1 \\ 0\\ 2 \\ 3 \\ 0 \end{array} \right] ;
\left[ \begin{array}{c} 0 \\ 2 \\ 0\\ 7 \\ 1 \\ 0 \end{array} \right] ;
\left[ \begin{array}{c} 0 \\ 12 \\ 0\\ 23 \\ 35 \\ 0 \end{array} \right] \right\}$


\item
$\left\{\left[ \begin{array}{c} 1 \\ 1 \\ 1 \\ 1 \end{array} \right] ; 
\left[ \begin{array}{c} 1 \\ 1 \\ 2 \\ 1 \end{array} \right] ;
\left[ \begin{array}{c} 1 \\ 1 \\ 1 \\ 2 \end{array} \right] \right\}
$
\end{enumerate}




\edXabox{type="oldmultichoice" expect="B","C","D" options="A","B","C","D","E","F"}

\edXsolution{  }

\endedxproblem


\endedxvertical







\beginedxvertical{Summarizing}



\doedxvideo{Completing the Chart}{Pumr8h6X7Gs}





\endedxvertical

