

\beginedxvertical{Page One}

\beginedxtext{Preliminaries}





At the end of this sequence, and after some practice, you should be able to:

\begin{itemize}
\item Translate systems of linear equations into {\keyb{\bf augmented matrices}}.
\item Apply elementary {\keyb{\bf row operations}} to matrices.  
\end{itemize}



For time budgeting purposes, this sequence has X videos totaling X minutes, 
plus some questions.  




\endedxtext

\endedxvertical

\beginedxvertical{Introduction}



\doedxvideo{Systems of Linear Equations}{MNuDcFXUymE}



\beginedxproblem{Matrix entry}{\dpa1}

Consider the following system of linear equations:

\begin{eqnarray*}
10x_1 + 30x_2 & = & 10 \\
2x_1 + 6x_2 & = & 3 
\end{eqnarray*}

\begin{edXshowhide}{How to Enter a Matrix}

To enter the matrix $\left[ \begin{array}{ccc}
0&1&2 \\
1&2&3 \end{array} \right]$, type [[0,1,2],[1,2,3]].  (Type the same thing if you want to enter the augmented matrix $\left[ \begin{array}{cc:c}
0&1&2 \\
1&2&3 \end{array} \right]$.)

Use decimals only.  

\end{edXshowhide}

\begin{edXscript}
def MatrixEntry(expect, ans):
  	import ast
	import numpy as np 
	ret= {'ok':False}
  	atol = 0.01
  	try:
		list_ans = ast.literal_eval(ans)
		list_expect = ast.literal_eval(expect)
  		matrix_ans = np.matrix(list_ans)
  		matrix_expect = np.matrix(list_expect) 
  		if matrix_ans.shape != matrix_expect.shape:
  			ret['msg'] = 'Wrong shape of matrix'
  		elif np.allclose(matrix_ans, matrix_expect,0.01,1e-08):
  			ret['ok'] = True
  		else:
  			ret['msg'] = 'Something is wrong'
	except SyntaxError:
		ret['msg'] = 'Wrong input format'
  	return ret
\end{edXscript}





Enter the coefficient matrix for the above system.  

To enter a matrix, type...


\edXabox{expect="Placeholder" options="Placeholder"}


Enter the augmented matrix for the above system. 


\edXabox{expect="Placeholder" options="Placeholder"}

\edXsolution{ 
}


\endedxproblem





\endedxvertical





\beginedxvertical{Doing Row Operations}

\doedxvideo{Row Operations}{y7WkHWviQUQ}



\beginedxproblem{Two Row Operations}{\dpa1}


Recall the system of equations from the previous question.  



\begin{eqnarray*}
10x_1 + 30x_2 & = & 10 \\
2x_1 + 6x_2 & = & 3 
\end{eqnarray*}





Do the following row operations to the augmented matrix.  First, multiply the first row by 
$\frac{1}{10}$.   Then, replace the second row with the second
row minus twice the first row.  What matrix do you get as a result?  

\begin{edXshowhide}{How to Enter a Matrix}

To enter the matrix $\left[ \begin{array}{ccc}
0&1&2 \\
1&2&3 \end{array} \right]$, type [[0,1,2],[1,2,3]].  (Type the same thing if you want to enter the augmented matrix $\left[ \begin{array}{cc:c}
0&1&2 \\
1&2&3 \end{array} \right]$.)

Use decimals only.  

\end{edXshowhide}

\begin{edXscript}
def MatrixEntry(expect, ans):
  	import ast
	import numpy as np 
	ret= {'ok':False}
  	atol = 0.01
  	try:
		list_ans = ast.literal_eval(ans)
		list_expect = ast.literal_eval(expect)
  		matrix_ans = np.matrix(list_ans)
  		matrix_expect = np.matrix(list_expect) 
  		if matrix_ans.shape != matrix_expect.shape:
  			ret['msg'] = 'Wrong shape of matrix'
  		elif np.allclose(matrix_ans, matrix_expect,0.01,1e-08):
  			ret['ok'] = True
  		else:
  			ret['msg'] = 'Something is wrong'
	except SyntaxError:
		ret['msg'] = 'Wrong input format'
  	return ret
\end{edXscript}



\edXabox{expect="Placeholder" options="Placeholder"}

\edXsolution{ 


}

\endedxproblem



\beginedxproblem{System Consistency?}{\dpa1}

Note the last row in your result, and think about what equation it represents.  
Is the system consistent?  If so, does it have a unique solution?  

\edXabox{type="multichoice" expect="The system is inconsistent" options="The system is consistent and has a unique solution","The system is consistent but has more than one solution","The system is inconsistent"}

\edXsolution{ 
}

\endedxproblem

\beginedxproblem{Yes or No}{\dpa2}

Does a linear system of two equations in two variables always have at least one solution?

\edXabox{expect="No" options="Yes","No"}

\edXsolution{  }

\endedxproblem




\beginedxproblem{More Row Operations}{\dpa1}


Consider this system of equations.  



\begin{eqnarray*}
0x_1 + 1x_2 + 1x_3 & = & 2 \\
1x_1 + 2x_2 + 1x_3 & = & 3 \\
2x_1 + 6x_2 + 4x_3 & = & 10 \\
\end{eqnarray*}





Do the following row operations to the augmented matrix.  First, switch the first two
rows.  Then replace the third row with the third
row minus twice the first row.  Next, replace the first row with the first
row minus twice the second row.  Finally, replace the third row with the third
row minus twice the second row.  What matrix do you get as a result?  

\begin{edXshowhide}{How to Enter a Matrix}

To enter the matrix $\left[ \begin{array}{ccc}
0&1&2 \\
1&2&3 \end{array} \right]$, type [[0,1,2],[1,2,3]].  (Type the same thing if you want to enter the augmented matrix $\left[ \begin{array}{cc:c}
0&1&2 \\
1&2&3 \end{array} \right]$.)

Use decimals only.  

\end{edXshowhide}

\begin{edXscript}
def MatrixEntry(expect, ans):
  	import ast
	import numpy as np 
	ret= {'ok':False}
  	atol = 0.01
  	try:
		list_ans = ast.literal_eval(ans)
		list_expect = ast.literal_eval(expect)
  		matrix_ans = np.matrix(list_ans)
  		matrix_expect = np.matrix(list_expect) 
  		if matrix_ans.shape != matrix_expect.shape:
  			ret['msg'] = 'Wrong shape of matrix'
  		elif np.allclose(matrix_ans, matrix_expect,0.01,1e-08):
  			ret['ok'] = True
  		else:
  			ret['msg'] = 'Something is wrong'
	except SyntaxError:
		ret['msg'] = 'Wrong input format'
  	return ret
\end{edXscript}



\edXabox{expect="Placeholder" options="Placeholder"}

\edXsolution{ 
}

\endedxproblem



\beginedxproblem{System Consistency?}{\dpa1}

Is the system consistent?  If so, does it have a unique solution?  

\edXabox{type="multichoice" expect="The system is consistent but has more than one solution" options="The system is consistent and has a unique solution","The system is consistent but has more than one solution","The system is inconsistent"}

\edXsolution{ 
}

\endedxproblem

\endedxvertical

\beginedxvertical{A Few Questions}


\beginedxproblem{Row of All Zeros}{\dpa2}

Suppose you have an augmented matrix and, after a few elementary row operations, you see
one row that looks like this: 

$\left[ \begin{array}{cccc:c} 0 & 0 & 0 & 0 & 0 \end{array} \right]$

Just from the presence of that one row, what can you conclude about the overall system?  

\edXabox{type="multichoice" expect="We cannot conclude anything; it will depend on the other rows" options="The system is consistent and has a unique solution","The system is consistent but has more than one solution","The system is inconsistent","We cannot conclude anything; it will depend on the other rows"}

\edXsolution{  

}

\endedxproblem



\beginedxproblem{Types of Row Operations}{\dpa2}

How many types of elementary row operations are there?


\edXabox{type="numerical" expect="3"}

\edXsolution{  }

\endedxproblem




\endedxvertical

\beginedxvertical{Elementary Row Operations}

\beginedxtext{Elementary Row Operations}


There are three types of elementary row operations that one can apply to a matrix.  

\begin{itemize}
\item Switch two rows.  
\item Multiply a row by a {\underline{non-zero}} scalar.  
\item Add a scalar multiple of one row to another row.   (More precisely, replace one row with the sum of itself and a scalar multiple of a different row.)  
\end{itemize}

If one can obtain augmented matrix $B$ from augmented matrix $A$ via a series of elementary row operations, 
we say that $A$ and $B$ are {\keyb{\bf row-equivalent.}} 

If $A$ and $B$ are row-equivalent, the system of linear equations represented by $A$ and the system of linear equations represented by $B$ have the same set of solutions.  

\endedxtext



\endedxvertical

