

\beginedxvertical{Page One}

\beginedxtext{Preliminaries}





At the end of this sequence, and after some practice, you should be able to:

\begin{itemize}
\item Use elementary {\keyb{\bf row operations}} to put matrices into {\keyb{\bf row-reduced echelon form (RREF)}}.
\item Interpret RREF matrices to determine {\keyb{\bf existence and uniqueness}} of solutions.  
\end{itemize}


For time budgeting purposes, this sequence has X videos totaling X minutes, 
plus some questions.  




\endedxtext

\endedxvertical


\beginedxvertical{Row-Reduced Echelon Form}

\doedxvideo{RREF}{Qi3O6iLhZ0E}



\endedxvertical


\beginedxvertical{RREF Defined}

\beginedxtext{Row-Reduced Echelon Form}


{\keya{\bf{Definition.}}}  A matrix is in Row-Reduced Echelon Form (RREF) if it satisfies the following conditions:

\begin{itemize}
\item All rows of all zeros are at the bottom.
\item In any non-zero row, the first non-zero entry is 1.  (This is called a {\keyb{\bf pivot}}.) 
\item Each pivot is the only non-zero entry in its column.
\item A pivot in a lower row is to the right of any pivot in a higher row.  
\end{itemize}



\endedxtext


\beginedxproblem{RREF or not?}{\dpa2}

Consider the following matrices.

\[ 
A = \left[ \begin{array}{ccccc} 1 & 9 & 1 & 0 & 0 \\
0 & 0 & 0 & 1 & 0 \\
0 & 0 & 0 & 0 & 0 \\
0 & 0 & 0 & 0 & 0 
 \end{array} \right], 

B= \left[ \begin{array}{ccccc} 1 & 0 & 3 & 1 & 0 \\
0 & 1 & 3 & 0 & 0 \\
0 & 0 & 0 & 1 & 0 \\
0 & 0 & 0 & 0 & 1 
 \end{array} \right] \]
 
\[ 
C = \left[ \begin{array}{ccccc} 1 & 9 & 0 & 0 & 0 \\
0 & 0 & -1 & 1 & 0 \\
0 & 0 & 0 & 0 & 1 \\
0 & 0 & 0 & 0 & 0
 \end{array} \right], 

D= \left[ \begin{array}{ccccc} 0 & 1 & 3 & 0 & 0 \\
0 & 0 & 0 & 1 & 0 \\
0 & 0 & 0 & 0 & 1 \\
0 & 0 & 0 & 0 & 0 
 \end{array} \right] \]

\[ 
E = \left[ \begin{array}{ccccc} 1 & 1 & 0 & 0 & 0 \\
0 & 0 & 1 & 0 & 0 \\
0 & 0 & 0 & 1 & 0 \\
0 & 0 & 0 & 0 & 1
 \end{array} \right], 

F= \left[ \begin{array}{ccccc} 0 & 1 & 2 & 0 & 0 \\
1 & 0 & 0 & 1 & 0 \\
0 & 0 & 0 & 0 & 1 \\
0 & 0 & 0 & 0 & 0 
 \end{array} \right] \] 
 
Which of them are in RREF?  Click all that apply.  




\edXabox{type="oldmultichoice" expect="A","D","E"  options="A","B","C","D","E","F"}

\edXsolution{Matrix $B$ has a pivot in row 3, column 4, but it is not the only non-zero entry in its column.

The first non-zero entry in the second row of matrix $C$ is not 1.  

The pivots in matrix $F$ do not go down and right. }

\endedxproblem



\beginedxproblem{Make it RREF}{\dpa2}

The following matrix is not in RREF.


$\left[ \begin{array}{ccccc} 1 & 4 & 0 & 0 & 7 \\
0 & 0 & 1 & -2 & 3 \\
0 & 0 & 0 & 1 & 1
 \end{array} \right]$

However, you can apply a single row operation to put it into RREF.  What matrix is the
result?  


\begin{edXshowhide}{How to Enter a Matrix}

To enter the matrix $\left[ \begin{array}{ccc}
0&1&2 \\
1&2&3 \end{array} \right]$, type [[0,1,2],[1,2,3]].  (Type the same thing if you want to enter the augmented matrix $\left[ \begin{array}{cc:c}
0&1&2 \\
1&2&3 \end{array} \right]$.)

Use decimals only.  

\end{edXshowhide}

\begin{edXscript}
def MatrixEntry(expect, ans):
  	import ast
	import numpy as np 
	ret= {'ok':False}
  	atol = 0.01
  	try:
		list_ans = ast.literal_eval(ans)
		list_expect = ast.literal_eval(expect)
  		matrix_ans = np.matrix(list_ans)
  		matrix_expect = np.matrix(list_expect) 
  		if matrix_ans.shape != matrix_expect.shape:
  			ret['msg'] = 'Wrong shape of matrix'
  		elif np.allclose(matrix_ans, matrix_expect,0.01,1e-08):
  			ret['ok'] = True
  		else:
  			ret['msg'] = 'Something is wrong'
	except SyntaxError:
		ret['msg'] = 'Wrong input format'
  	return ret
\end{edXscript}



\edXabox{type="custom" cfn="MatrixEntry" expect="[[1,4,0,0,7],[0,0,1,0,5],[0,0,0,1,1]]"}

\edXsolution{Note that the matrix currently satisfies all conditions except ``Each pivot is the only non-zero entry in its column", for there is a $-2$ above the pivot in the third row.  We can add two times row three to row two and replace row 2 with the result.  Doing so, we obtain

$\left[ \begin{array}{ccccc} 1 & 4 & 0 & 0 & 7 \\
0 & 0 & 1 & 0 & 5 \\
0 & 0 & 0 & 1 & 1
 \end{array} \right].$}

\endedxproblem



\beginedxproblem{Pivots vs. Rows}{\dpa2}

True or False: If a matrix that is in RREF has 6 rows, then it must have 6 pivots.  


\edXabox{expect="False" options="True","False"}

\edXsolution{In fact, a matrix with $6$ rows may have any number of pivots between $0$ and $6$.  For example, the $6\times 6$ zero matrix has 6 rows, and zero pivots, while the $6\times 6$ matrix  $\left[ \begin{array}{cccccc} 
 1 & 0 & 0 & 0 & 0&0 \\
 0 & 1 & 0 & 0 & 0&0 \\
 0 & 0 & 1 & 0 & 0&0 \\
 0 & 0 & 0 & 1 & 0&0 \\
 0 & 0 & 0 & 0 & 1&0 \\
 0 & 0 & 0 & 0 & 0&0 \\
 \end{array} \right]$ has 6 rows and 5 pivots.}

\endedxproblem

\endedxvertical


\beginedxvertical{Row Reduction Algorithm}

\doedxvideo{The Row Reduction Algorithm}{oXO9Iw9KyPo}



\beginedxtext{The Row-Reduction Algorithm}

The following algorithm will always work to reduce a matrix into Row-Reduced Echelon Form.  

\begin{itemize}
\item Swap two rows to get the left-most remaining non-zero entry to the topmost remaining row.  
\item Scale that row to make its leading entry a 1.    
\item Clear all non-zero entries in the column of that pivot (both above and below the pivot) by adding/subtracting multiples of that row to/from other rows.  
\item With that pivot now set, repeat the procedure with the remaining rows.  
\end{itemize}

This is sometimes called {\keyb{\bf  Gaussian elimination}}.  

\endedxtext



\endedxvertical


\beginedxvertical{Row-Reducing Practice}


\beginedxproblem{Row-reduction Practice}{\dpa1}

Let 
\[
A = \left[ \begin{array}{ccc} 0 & 1 & 2 \\ 3 & 0 & -3 \\ 2 & 2 & 5 \\ 0 & -2 & 1\end{array} \right]. \]

What matrix do you get if you row-reduce $A$? 

\begin{edXscript}
def MatrixEntry(expect, ans):
  	import ast
	import numpy as np 
	ret= {'ok':False}
  	atol = 0.01
  	try:
		list_ans = ast.literal_eval(ans)
		list_expect = ast.literal_eval(expect)
  		matrix_ans = np.matrix(list_ans)
  		matrix_expect = np.matrix(list_expect) 
  		if matrix_ans.shape != matrix_expect.shape:
  			ret['msg'] = 'Wrong shape of matrix'
  		elif np.allclose(matrix_ans, matrix_expect,0.01,1e-08):
  			ret['ok'] = True
  		else:
  			ret['msg'] = 'Something is wrong'
	except SyntaxError:
		ret['msg'] = 'Wrong input format'
  	return ret
\end{edXscript}


\edXabox{type="custom" cfn="MatrixEntry" expect="[[1,0,0],[0,1,0],[0,0,1],[0,0,0]]"}

\edXsolution{ \begin{itemize}
\item Swap two rows to get the left-most remaining non-zero entry to the topmost remaining row.  
\item Scale that row to make its leading entry a 1.    
\item Clear all non-zero entries in the column of that pivot (both above and below the pivot) by adding/subtracting multiples of that row to/from other rows.  
\item With that pivot now set, repeat the procedure with the remaining rows.  
\end{itemize}

\[
A = \left[ \begin{array}{ccc} 0 & 1 & 2 \\ 3 & 0 & -3 \\ 2 & 2 & 5 \\ 0 & -2 & 1\end{array} \right]. \]

\begin{itemize}
    \item Swap R1 and R2:
    \[\left[ \begin{array}{ccc} 3 & 0 & -3 \\ 0 & 1 & 2 \\ 2 & 2 & 5 \\ 0 & -2 & 1\end{array} \right]. \]
    
    \item Scale the new R1 by $\frac{1}{3}$:
     \[\left[ \begin{array}{ccc} 1 & 0 & -1 \\ 0 & 1 & 2 \\ 2 & 2 & 5 \\ 0 & -2 & 1\end{array} \right]. \]
     
     \item Subtract twice R1 from R3 and replace R3 to clear the non-zero entry below the pivot in the first row:
     \[\left[ \begin{array}{ccc} 1 & 0 & -1 \\ 0 & 1 & 2 \\ 0 & 2 & 7 \\ 0 & -2 & 1\end{array} \right]. \]
     
    \item We focus on the remaining rows.  The left-most nonzero entries are in row 2 and 3, and the leading entry of the second row is already 1, so we do not need to perform any swap or scaling to satisfy the conditions in the first and second step of the algorithm.\\
    Now, we clear the non-zero entry in the third row below this pivot by replacing R3 with R3$-2\cdot$R2.  We also clear the non-zero entry in the fourth row below this pivot by replacing R4 with R4$+2\cdot $R2.
     \[\left[ \begin{array}{ccc} 1 & 0 & -1 \\ 0 & 1 & 2 \\ 0 & 0 & 3 \\ 0 & 0 & 5\end{array} \right]. \]
     \item Now, the first left-most non-zero entry is already in the topmost remaining row (row 3), so we scale it by $\frac{1}{3}$ to obtain
     \[\left[ \begin{array}{ccc} 1 & 0 & -1 \\ 0 & 1 & 2 \\ 0 & 0 & 1 \\ 0 & 0 & 5\end{array} \right]. \]
     \item Finally, we clear the non-zero entries above and below that pivot by replacing R1 with R1$+$R3, replacing R2 with R2$-2\cdot$R3 and replacing R4 with R4$-5\cdot$R3.
     \[\left[ \begin{array}{ccc} 1 & 0 & 0 \\ 0 & 1 & 0 \\ 0 & 0 & 1 \\ 0 & 0 & 0\end{array} \right]. \]
     
\end{itemize}
We can then check that this final matrix satisfies the conditions of RREF.  Applying the algorithm to this final matrix will not require any row operations.
}


\endedxproblem


\beginedxproblem{Linear System 1}{\dpa1}

Consider the augmented matrix
\[
A = \left[ \begin{array}{ccc:c} 0 & 1 & 2 & -1 \\ 3 & 0 & -3 & 3 \\ 
2 & 2 & 5 & 3\\ 0 & -2 & 1 & 7\end{array} \right]. \]


Note that the coefficient matrix is the same as the matrix $A$ from the previous question.

Applying the row-reduction algorithm to this augmented matrix, what can we conclude about
the system?  

\edXabox{type="multichoice" expect="The system is consistent and has a unique solution" options="The system is consistent and has a unique solution","The system is consistent but has more than one solution","The system is inconsistent"}

\edXsolution{After repeating the first 3 steps from above to this matrix, we obtain
\[
A = \left[ \begin{array}{ccc:c} 1 & 0 & -1 & 1 \\ 0 & 1 & 2 & -1 \\ 
0 & 2 & 7 & 1\\ 0 & -2 & 1 & 7\end{array} \right]. \].

Repeating step 4, we obtain

\[
A = \left[ \begin{array}{ccc:c} 1 & 0 & -1 & 1 \\ 0 & 1 & 2 & -1 \\ 
0 & 0 & 3 & 3\\ 0 & 0 & 5 & 5\end{array} \right]. \].

Repeating step 5, we obtain
\[
A = \left[ \begin{array}{ccc:c} 1 & 0 & -1 & 1 \\ 0 & 1 & 2 & -1 \\ 
0 & 0 & 1 & 1\\ 0 & 0 & 5 & 5\end{array} \right]. \].

And, finally, repeating step 6, we obtain

\[
A = \left[ \begin{array}{ccc:c} 1 & 0 & 0 & 2 \\ 0 & 1 & 0 & -3 \\ 
0 & 0 & 1 & 1\\ 0 & 0 & 0 & 0\end{array} \right]. \].

This represents the equations $x_1 = 2$, $x_2 = -3$, $x_3 = 1$, and $0=0$.  Thus there is a unique
solution $(x_1, x_2, x_3) = (2,-3,1)$.  
}


\endedxproblem



\endedxvertical




\beginedxvertical{Interpretations}

\doedxvideo{Interpreting a RREF matrix}{cdAGco8RsOI}





\beginedxproblem{Matrix 1}{\dpa1}

Suppose you do some row operations to an augmented matrix, and the result is the following augmented
matrix:
\[
\left[ \begin{array}{cc:c} 1 & -3 & -2 \\ 0 & 0 & 1 \\ 0 & 0 & 0 \end{array} \right] \]

How many free variables are there?
\edXabox{type="numerical" expect="1"}

What can you conclude about the original system of equations?  

\edXabox{type="multichoice" expect="The system is inconsistent" options="The system is consistent and has a unique solution","The system is consistent but has more than one solution","The system is inconsistent"}

\edXsolution{The second row corresponds to an equation of the form $0x+0y=1$, which has no solutions.  Thus, the original system will also have no solutions.  This is true even though the second column has no pivot
and therefore represents a free variable.  
}

\endedxproblem

\beginedxproblem{Matrix 2}{\dpa1}

Suppose you do some row operations to an augmented matrix, and the result is the following augmented
matrix:
\[  
\left[ \begin{array}{cc:c} 1 & 0 & 0 \\ 0 & 1 & 1 \\ 0 & 0 & 0 \end{array} \right] \]


How many free variables are there?
\edXabox{type="numerical" expect="0"}

What can you conclude about the original system of equations?  

\edXabox{type="multichoice" expect="The system is consistent and has a unique solution" options="The system is consistent and has a unique solution","The system is consistent but has more than one solution","The system is inconsistent"}

\edXsolution{The original system corresponds to 3 equations in 2 variables.  Since there are two pivot positions in the RREF and 2 variables, we can conclude that there are $2-2=0$ free variables.  Thus, the original system is consistent and has a unique solution.
}

\endedxproblem


\beginedxproblem{Matrix 3}{\dpa1}

Suppose you do some row operations to an augmented matrix, and the result is the following augmented
matrix:
\[ 
\left[ \begin{array}{cccc:c} 1 & 0 & 0 & 1 & -2 \\ 0 & 0 & 1 & 3 & 3 \\ 0 & 0 & 0 & 0 & 0 \end{array} \right] \]


How many free variables are there?
\edXabox{type="numerical" expect="2"}



What can you conclude about the original system of equations?  

\edXabox{type="multichoice" expect="The system is consistent but has more than one solution" options="The system is consistent and has a unique solution","The system is consistent but has more than one solution","The system is inconsistent"}

\edXsolution{The original system corresponds to 3 equations in 4 variables.  Since there are 2 pivot positions in the RREF, there are $4-2=2$ free variables.  Since there is no row of zeros followed by a non-zero entry,
the system is consistent, but the existence of free variables ensures that any solution to the system is not unique.
}

\endedxproblem


\beginedxproblem{Free Variables}{\dpa1}

If a system of linear equations has free variables when row-reduced, is it possible for the system to be inconsistent?  

\edXabox{expect="Yes" options="Yes","No"}

\edXsolution{We previously saw a system of 3 linear equations in 2 variables with the following augmented matrix in RREF: \[
\left[ \begin{array}{cc:c} 1 & -3 & -2 \\ 0 & 0 & 1 \\ 0 & 0 & 0 \end{array} \right]. \]  There is no pivot in the column corresponding to $x_2$.  Thus, there is a free variable.  However, the second row indicates that there are no solutions, and the system is inconsistent.   
}

\endedxproblem

\endedxvertical


\beginedxvertical{Finding Solutions}




\beginedxproblem{Solutions to a System}{\dpa8}

Recall the augmented matrix from the last problem:

\[ 
\left[ \begin{array}{cccc:c} 1 & 0 & 0 & 1 & -2 \\ 0 & 0 & 1 & 3 & 3 \\ 0 & 0 & 0 & 0 & 0 \end{array} \right] \]

Working from right to left as in the last video, what are all possible solutions to this system?
Type the word `free' (lower-case) if the variable is a free variable.  For others, remember to use * for times and type 
xn for the variable $x_n$.  For instance, to enter $-3x_2 + 2x_4$, type -3*x2 + 2*x4.  

\edXinline{$x_4 = $}\edXabox{type="formula" expect="free" samples="free,x3@1,1:4,4#10" tolerance=".001" feqin="1" inline="1"}

\edXinline{$x_3 = $}\edXabox{type="formula" expect="3-3*x4" samples="free,x4,x2,x1@1,1,1,1:4,4,4,4#10" tolerance=".001" feqin="1" inline="1"}

\edXinline{$x_2 = $}\edXabox{type="formula" expect="free" samples="free,x4,x3,x1@1,1,1,1:4,4,4,4#10" tolerance=".001" feqin="1" inline="1"}

\edXinline{$x_1 = $}\edXabox{type="formula" expect="-2-x4" samples="free,x4,x3,x2@1,1,1,1:4,4,4,4#10" tolerance=".001" feqin="1" inline="1"}


\edXsolution{Working from right to left, we first see that there is not a pivot in the column corresponding the $x_4$.  Thus, $x_4$ is free.  We then see a pivot in the column corresponding to $x_3$, and corresponding equaion is $0x_1+0x_2+1x_3+3x_4=3$.  This gives $x_3=3-3x_4$.  Next, there is not a pivot in the column corresponding to $x_2$, and thus $x_2$ is free.   Finally, there is a pivot in the column corrsponding to $x_1$ and the corresponding equation is $1x_1+0x_2+0x_3+1x_4=-2$.  This gives $x_1=-2-x_4$.
}

\endedxproblem


\endedxvertical




\beginedxvertical{Going Further}



\beginedxtext{The Size of a Matrix}

If a matrix $M$ has $m$ rows and $n$ columns, we say that 
$M$ is an $m\times n$ matrix.  For instance, 

\[ 
\left[ \begin{array}{cccc:c} 1 & 0 & 0 & 1 & -2 \\ 0 & 0 & 1 & 3 & 3 \\ 0 & 0 & 0 & 0 & 0 \end{array} \right] \]

is a $3\times 5$ augmented matrix, since it has 3 rows and 5 columns.  The coefficient matrix of the above system would be a $3\times 4$ matrix.  


\endedxtext




\beginedxproblem{Matrix Size}{\dpa1}

Let 
\[
A = \left[ \begin{array}{ccc} 1 & -3 & -2 \\ 0 & 0 & 1 \\ 0 & 0 & 0 \\ 0 & 0 & 0\end{array} \right]. \]

What is the size of $A$? 
\edXabox{type="multichoice" expect="$4\times 3$" options="$3\times 4$","$4\times 3$"}

\edXsolution{We see that $A$ has 4 rows and 3 columns.  The first entry of $m\times n$ corresponds to the number of rows, while the second corresponds to number of columns. 
}

\endedxproblem


\beginedxproblem{Maximum pivots}{\dpa1}

What is the maximum number of pivots in a $6\times 3$ coefficient matrix, when row-reduced?  

\edXabox{type="numerical" expect="3"}


\edXsolution{There aren't any non-zero entries above or below a pivot.  Thus, there is at most one pivot in each column.  Thus there can be no more than 3 pivots.  There are enough rows that this is possible to attain.    
}

\endedxproblem

\beginedxproblem{Maximum pivots 2}{\dpa1}

What is the maximum number of pivots in a $4\times 6$ coefficient  matrix, when row-reduced?  

\edXabox{type="numerical" expect="4"}


\edXsolution{By definition, there is at most one pivot in each row.  So there can be at most 4 pivots.  There
are enough columns that this is possible to attain.  
}

\endedxproblem

\beginedxproblem{Number of Free Variables}{\dpa1}

What is the strongest statement you can make about the number of free variables in a $4\times 6$ coefficient  matrix, when row-reduced?  

\edXinline{The number of free variables must be  }\edXabox{expect="at least" options="exactly","at least","at most" inline="1"} \edXabox{type="numerical" expect="2" inline="1"}

\edXsolution{We know that a $4\times 6$ coefficient matrix corresponds to a system of equations with 6 variables.  By the previous problem, we know that there are at most 4 pivots.  Thus, there are at least $6-4=2$ free variables.  Now, we can have as many as 6 free variables (consider the zero matrix as a coefficient matrix).  You can come up with examples of matrices with any number between 0 and 4 pivots, and thus any number between 2 and 6 free variables.
}

\endedxproblem

\beginedxproblem{4 Equations, 6 Variables}{\dpa1}

Given the answer to the previous question, which of the following results are possible for a system of 4 linear equations in 6 variables?

\edXabox{type="oldmultichoice" expect="The system is consistent but has more than one solution","The system is inconsistent" options="The system is consistent and has a unique solution","The system is consistent but has more than one solution","The system is inconsistent"}

\edXsolution{The existence of free variables rules out the possibility for unique solutions.  But, we know that a system with free variables may be consistent or inconsistent.
}

\endedxproblem



\endedxvertical


\beginedxvertical{Fields}



\beginedxproblem{Real Numbers}{\dpa1}

If a matrix has real number entries, will the entries stay real numbers when you 
row-reduce it?  

\edXabox{expect="Yes" options="Yes","Not necessarily"}

\edXsolution{Yes.  The only acceptable row operations are multiplication by a scalar, addition, and subraction.  In the case of a matrix with real entries, row-reducing will not require a non-real scalar.   The sum, product and difference of two real numbers is also real.  Thus, under each of these operations, the entries will remain real.
}

\endedxproblem

\beginedxproblem{Rational Numbers}{\dpa1}

If a matrix has rational number entries, will the entries stay rational numbers when you 
row-reduce it?  (Recall that rational numbers are those that can be written as fractions
$\frac{p}{q}$,
where $p,q$ are integers.)  

\edXabox{expect="Yes" options="Yes","Not necessarily"}

\edXsolution{Yes.  The only acceptable row operations are multiplication by a scalar, addition, and subraction.  In the case of a matrix with rational entries, row-reducing will not require a non-rational scalar.   The sum, product and difference of two rational numbers is also rational.  Thus, under each of these operations, the entries will remain rational.
}

\endedxproblem

\beginedxproblem{Integers}{\dpa1}

If a matrix has integer entries, will the entries stay integers when you 
row-reduce it?  

\edXabox{expect="Not necessarily" options="Yes","Not necessarily"}

\edXsolution{No.  For example, to put  \[\left[ \begin{array}{ccc} 2 & 0 & 1 \\ 0 & 1 & 1 \end{array} \right]. \] into RREF, we have to multiply Row 1 by $\frac{1}{2}$, which will create a non-integer entry. 
}

\endedxproblem


\doedxvideo{Fields}{3e74wlqpAf4}


\endedxvertical


\beginedxvertical{Fields Defined}


\beginedxtext{Fields}

A {\keyb{\bf field}} is, essentially, a set in which one can add, subtract, multiply, and
divide any two elements (other than dividing by zero).  The formal definition is below.  

The field that we will
most commonly use in this course is the set of real numbers $\R$.  Other fields include
the set of complex numbers $\C$, the set of rational numbers $\Q$, and the integers modulo
a prime $p$.  

\begin{edXshowhide}{Formal Definition of a Field}
A field is a set $F$ together with binary operations $+, \cdot$ on $F$ which satisfy the
following conditions:

\begin{itemize}
\item Addition is associative and commutative; that is, for all $a,b,c \in F$, we have
$a+(b+c) = (a+b)+c$ and $a+b  = b+a$.  
\item There is an element $0 \in F$ such that $a+0 = a$ for all $a \in F$.
\item For every $a\in F$, there is an element $-a \in F$ such that $-a + a = 0$.  
\item Multiplication is associative and commutative; that is, for all $a,b,c \in F$, we have
$a(bc) = (ab)c$ and $ab  = ba$.  
\item There is a non-zero element $1 \in F$ such that $1a = a$ for all $a \in F$.  
\item For every non-zero $a\in F$, there is an element $a\inv$ such that $a\inv a = 1$.  
\item Multiplication distributes over addition; that is, for all $a,b,c \in F$, $a(b+c) = ab + ac$.  
\end{itemize}

\end{edXshowhide}




\endedxtext

\endedxvertical
