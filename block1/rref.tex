

\beginedxvertical{Page One}

\beginedxtext{Preliminaries}





At the end of this sequence, and after some practice, you should be able to:

\begin{itemize}
\item Use elementary {\keyb{\bf row operations}} to put matrices into {\keyb{\bf row-reduced echelon form (RREF)}}.
\item Interpret RREF matrices to determine {\keyb{\bf existence and uniqueness}} of solutions.  
\end{itemize}


For time budgeting purposes, this sequence has X videos totaling X minutes, 
plus some questions.  




\endedxtext

\endedxvertical


\beginedxvertical{Row-Reduced Echelon Form}

\doedxvideo{RREF}{Qi3O6iLhZ0E}



\endedxvertical


\beginedxvertical{RREF Defined}

\beginedxtext{Row-Reduced Echelon Form}


A matrix is in Row-Reduced Echelon Form (RREF) if it satisfies the following conditions:

\begin{itemize}
\item All rows of all zeros are at the bottom.
\item In any non-zero row, the first non-zero entry is 1.  (This is called a {\keyb{\bf pivot}}.) 
\item Each pivot is the only non-zero entry in its column.
\item A pivot in a lower row is to the right of any pivot in a higher row.  
\end{itemize}



\endedxtext


\beginedxproblem{RREF or not?}{\dpa2}

Which of the following matrices are in RREF?  Click all that apply.  



\edXabox{type="oldmultichoice" expect="mx1" options="mx1","mx2"}

\edXsolution{  }

\endedxproblem



\beginedxproblem{Make it RREF}{\dpa2}

The following matrix is not in RREF.


$\left[ \begin{array}{ccccc} 1 & 4 & 0 & 0 & 7 \\
0 & 0 & 1 & -2 & 3 \\
0 & 0 & 0 & 1 & 1
 \end{array} \right]$

However, you can apply a single row operation to put it into RREF.  What matrix is the
result?  


\begin{edXshowhide}{How to Enter a Matrix}

To enter the matrix $\left[ \begin{array}{ccc}
0&1&2 \\
1&2&3 \end{array} \right]$, type [[0,1,2],[1,2,3]].  (Type the same thing if you want to enter the augmented matrix $\left[ \begin{array}{cc:c}
0&1&2 \\
1&2&3 \end{array} \right]$.)

Use decimals only.  

\end{edXshowhide}

\begin{edXscript}
def MatrixEntry(expect, ans):
  	import ast
	import numpy as np 
	ret= {'ok':False}
  	atol = 0.01
  	try:
		list_ans = ast.literal_eval(ans)
		list_expect = ast.literal_eval(expect)
  		matrix_ans = np.matrix(list_ans)
  		matrix_expect = np.matrix(list_expect) 
  		if matrix_ans.shape != matrix_expect.shape:
  			ret['msg'] = 'Wrong shape of matrix'
  		elif np.allclose(matrix_ans, matrix_expect,0.01,1e-08):
  			ret['ok'] = True
  		else:
  			ret['msg'] = 'Something is wrong'
	except SyntaxError:
		ret['msg'] = 'Wrong input format'
  	return ret
\end{edXscript}



\edXabox{expect="Placeholder" options="Placeholder"}

\edXsolution{  }

\endedxproblem



\beginedxproblem{Pivots vs. Rows}{\dpa2}

True or False: If a matrix that is in RREF has 6 rows, then it must have 6 pivots.  


\edXabox{expect="False" options="True","False"}

\edXsolution{  }

\endedxproblem

\endedxvertical


\beginedxvertical{Interpretations}

\doedxvideo{The Row Reduction Algorithm}{oXO9Iw9KyPo}



\beginedxtext{The Row-Reduction Algorithm}

The following algorithm will always work to reduce a matrix into Row-Reduced Echelon Form.  

\begin{itemize}
\item Swap two rows to get the left-most remaining non-zero entry to the topmost remaining row.  
\item Scale that row to make its leading entry a 1.    
\item Clear all non-zero entries in the column of that pivot (both above and below the pivot) by adding/subtracting multiples of that row to/from other rows.  
\item With that pivot now set, repeat the procedure with the remaining rows.  
\end{itemize}

This is sometimes called {\keyb{\bf  Gaussian elimination}}.  

\endedxtext



\endedxvertical


\beginedxvertical{Row-Reducing Practice}


\beginedxproblem{Row-reduction Practice}{\dpa1}

Let 
\[
A = \left[ \begin{array}{ccc} 0 & 1 & 2 \\ 3 & 0 & -3 \\ 2 & 2 & 5 \\ 0 & -2 & 1\end{array} \right]. \]

What matrix do you get if you row-reduce $A$? 

\begin{edXshowhide}{How to Enter a Matrix}

To enter the matrix $\left[ \begin{array}{ccc}
0&1&2 \\
1&2&3 \end{array} \right]$, type [[0,1,2],[1,2,3]].  (Type the same thing if you want to enter the augmented matrix $\left[ \begin{array}{cc:c}
0&1&2 \\
1&2&3 \end{array} \right]$.)

Use decimals only.  

\end{edXshowhide}

\begin{edXscript}
def MatrixEntry(expect, ans):
  	import ast
	import numpy as np 
	ret= {'ok':False}
  	atol = 0.01
  	try:
		list_ans = ast.literal_eval(ans)
		list_expect = ast.literal_eval(expect)
  		matrix_ans = np.matrix(list_ans)
  		matrix_expect = np.matrix(list_expect) 
  		if matrix_ans.shape != matrix_expect.shape:
  			ret['msg'] = 'Wrong shape of matrix'
  		elif np.allclose(matrix_ans, matrix_expect,0.01,1e-08):
  			ret['ok'] = True
  		else:
  			ret['msg'] = 'Something is wrong'
	except SyntaxError:
		ret['msg'] = 'Wrong input format'
  	return ret
\end{edXscript}



\edXabox{expect="Placeholder" options="Placeholder"}

\edXsolution{ 
}


\endedxproblem


\beginedxproblem{Linear System 1}{\dpa1}

Consider the augmented matrix
\[
A = \left[ \begin{array}{ccc:c} 0 & 1 & 2 & -1 \\ 3 & 0 & -3 & 3 \\ 
2 & 2 & 5 & 3\\ 0 & -2 & 1 & 7\end{array} \right]. \]


Note that the coefficient matrix is the same as the matrix $A$ from the previous question.

Applying the row-reduction algorithm to this augmented matrix, what can we conclude about
the system?  

\edXabox{type="multichoice" expect="The system is consistent and has a unique solution" options="The system is consistent and has a unique solution","The system is consistent but has more than one solution","The system is inconsistent"}

\edXsolution{ 
}


\endedxproblem



\endedxvertical



