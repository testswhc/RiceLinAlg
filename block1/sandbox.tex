

\beginedxvertical{Page One}

\beginedxtext{Preliminaries}


For time budgeting purposes, this sequence has 3 videos totaling 11 minutes, 
plus some questions.  

Remember, when you're doing the online learning sequences, you may seek help if you 
do not understand a video, but you should think about all of the questions 
entirely individually.  You have pledged to do so under the Honor Code!  



\begin{eqnarray*}
2x_1 + 3x_2 & = & 1 \\
4x_1 + 6x_2 & = & 2 
\end{eqnarray*}

\endedxtext

\endedxvertical

\beginedxvertical{Rotating the Other Way}


\beginedxproblem{Rotate}{\dpa1}

\includesvg[400]{shellregion}


\begin{eqnarray*}
2x_1 + 3x_2 & = & 1 \\
4x_1 + 6x_2 & = & 2 
\end{eqnarray*}

Above is the graph of the region between the graph of $f(x) = x-x^3$ and the $x$ axis, for $0\le x \le 1$.  

We want to calculate the volume of the solid obtained by rotating this region around the
$y$ axis.  If we insist on calculating the volume using washers or disks, which way would
we have to slice the region?  

\edXabox{expect="Integrate with respect to y, using horizontal rectangles" options="Integrate with respect to x, using vertical rectangles","Integrate with respect to y, using horizontal rectangles","Both methods will yield
washers or disks"}

\edXsolution{ 
Horizontal rectangles, when rotated around the $y$ axis, yield washers (or disks).  If we
use vertical rectangles, they will form a different shape.  See the next video for more details!

}

\endedxproblem

\doedxvideo{Rotating}{XqYl95NKsw8}

\endedxvertical

\beginedxvertical{Shell Volume}


\beginedxproblem{Shell Dimensions}{\dpa1}

\includesvg[400]{shellregionrect}

When the region under the graph of $f(x) = x-x^3$ is rotated around the $y$ axis, the rectangle shown will 
rotate to form a cylindrical shell.  What will the thickness of the shell be?  


\edXabox{expect="Delta x" options="x","Delta x","f(x)","Something else"}

What will the height of the shell be?  

\edXabox{expect="f(x)" options="x","Delta x","f(x)","Something else"}

What will the radius of the shell be?  (Technically there are two radii, and we
are choosing the outer radius in this case, but remember that
since the shell is thin we can think of it as having a single radius as an approximation.)  

\edXabox{expect="x" options="x","Delta x","f(x)","Something else"}


\edXsolution{ 
The thickness of the shell comes from the thickness of the rectangle.  This is $\Delta x$.

The height of the shell comes from the height of the rectangle.  This is $f(x)$.  

The radius of the shell is the distance from the rectangle to the axis of rotation.  The rectangle has
(roughly) horizontal coordinate $x$, and the axis of rotation has horizontal coordinate 0, so 
the distance is $x$.  
}

\endedxproblem

\beginedxproblem{Shell Circumference}{\dpa2}

Given the answers to the previous question, what will the circumference of the shell be, in
terms of $x$?  

Type  * for multiplication, pi for $\pi$.  


\edXabox{type="formula" expect="2*pi*x" samples="x@1:2#5" feqin="1" tolerance=".01"}

\edXsolution{ The circumference is $2\pi$ times the radius. }

\endedxproblem



\doedxvideo{Volume with Shells}{Ctfn21CUE9o}

\endedxvertical

\beginedxvertical{Another Shell Volume}

\beginedxproblem{Different Axis}{\dpa2}

\includesvg[400]{shellregionrect2}

We continue to look at the region under the graph of $f(x) = x-x^3$, between $x=0$ and $x=1$, but now we rotate it around the line $x=1$.  The rectangle shown will 
rotate to form a cylindrical shell.  

What will the height of the shell be?  


\edXabox{type="formula" expect="x-x^3" samples="x@1:2#5" feqin="1" tolerance=".01"}

What will the radius of the shell be?  

\edXabox{type="formula" expect="1-x" samples="x@1:2#5" feqin="1" tolerance=".01"}


\edXsolution{ 
The height is the same as the height of the rectangle, so $x-x^3$.  

The radius is the horizontal distance from the rectangle to the axis of rotation.  The axis of rotation
is at $x=1$, and the rectangle has horizontal coordinate $x$, so the distance is
$1-x$.  
}

\endedxproblem

\beginedxproblem{Volume of a Shell}{\dpa2}


Using the ``unrolling" approximation, what will the volume of the shell be? 

Type Dx for $\Delta x$, pi for $\pi$, * for multiplication.  

\edXabox{type="formula" expect="2*pi*(1-x)*(x-x^3)*Dx" samples="x,Dx@1,1:2,2#5" feqin="1" tolerance=".01"}


\edXsolution{ 
The volume of the shell when ``unrolled" is approximated by its circumference times its height
times its thickness.  

This is $2\pi(1-x)(x-x^3)\Delta x$.  

}

\endedxproblem

\beginedxproblem{Add them up!}{\dpa2}


When we add the volumes of all the shells up, we get a Riemann sum.  Taking the limit as the number of slices
goes to infinity, we get an integral $\int_?^? k(x) \ dx.$  

What is the function $k(x)$?

\edXabox{type="formula" expect="2*pi*(1-x)*(x-x^3)" samples="x@1:2#5" feqin="1" tolerance=".01"}

What is the lower limit of integration?


\edXabox{type="numerical" expect="0"  feqin="1" tolerance=".01"}

What is the upper limit of integration?

\edXabox{type="numerical" expect="1"  feqin="1" tolerance=".01"}

\edXsolution{ 
Adding up and treating the total as a Riemann sum, we end up with
\[ \int_0^1 2\pi(1-x)(x-x^3)\ dx. \]
}

\endedxproblem

\endedxvertical

\beginedxvertical{One Last Example}

\beginedxproblem{Horizontal or Vertical}{\dpa1}

\includesvg[400]{yshellsexample}

Here is the region between the graphs $x = y^2$, $x+y = 6$, and $y = -1$.  We
wish to find the volume of the solid obtained by rotating this region around the line
$y = -1$.  

If we integrate with respect to $x$, the rectangles would become what shape when rotated?
 
\edXabox{expect="Disks or Washers" options="Disks or Washers","Shells","Something else"}

How many different integrals would we need to set up if we did that?  

\edXabox{type="numerical" expect="3"  feqin="1" tolerance=".01"}

If we integrate with respect to $y$, the rectangles would become what shape when rotated?
 
\edXabox{expect="Shells" options="Disks or Washers","Shells","Something else"}

How many different integrals would we need to set up if we did that?  

\edXabox{type="numerical" expect="1"  feqin="1" tolerance=".01"}


\edXsolution{ See the next video! }

\endedxproblem

\doedxvideo{Y Shells}{hqmX3tYO3ts}

% ask question - do you get shells or washers when blah is rotated?

\endedxvertical

